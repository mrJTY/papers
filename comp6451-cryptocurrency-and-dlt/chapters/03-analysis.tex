\section{Current state and analysis}

\subsection{Governance and path to decentralisation}
% Governance
Originally under Swirlds, Hashgraph's ownership was transferred to the Hedera Governing Council. It is a decentralised council of term-limited multinational companies \cite{baird2018hedera}. 

It's decentralised governance is based on Visa's\footnote{https://hedera.com/council} model which ensures that council does what is best for the network. No single company has complete control and any malicious acts will certainly damage a member's reputation. Some well-known names are in the council that includes companies like Google, Boeing, LG, and Eftpos Australia. 

A criticism of Hashgraph is that the network must agree on N, the total number of participants \cite{kauflin2018}. As of early 2021, it is currently on a permissioned system which makes N known. Citics also argue because of it's permissioned nature, it is not true to the roots of a fully decentralised network.

% It is still on a path towards decentralisation while ensuring security along the way.
% https://hedera.com/learning/what-is-hedera-hashgraph

\subsection{Security and tokenomics}
% Analysis on security
To be fully decentralised, Hashgraph's proof-of-stake system will rely on no 1/3 of the coins are owned by malicious actors. Hedera has outlined schedule towards decentralisation without compromising the network's security.

Compared with the proof-of-work system used by Bitcoin, proof-of-stake is more environmentally friendly as no mining is involved. It can also be argued to be fairer because it does not favour entities with economies of scale or access to cheap electricity. Hashgraph also has a coin release schedule to prevent whales to hoard the coins early.% \cite{coin-schedule}

\subsection{Performance analysis}
Blockhain-based systems still benefit from incumbency and familiarity \cite{khariff2018}. Gossip relies on assumptions to be fully robust \cite{alvisi}. Gossip is a well known protocol, but combining it with the concept of virtual voting is novel. As it is a new concept, the algorithm was checked by a Coq system (a formal verification system) which proves that it is aBFT - the highest security possible in a distributed system \cite{coq2018}.

Real world experiments have shown throughputs of 50-000-500,000+ transactions-per-second (depending on the region setup)\cite{baird2018hedera}. It still remains to be battle tested in a permissionless setup, but the current volumes (\url{https://hedera.com/dashboard}) are orders of magnitude to what Bitcoin or Ethereum can offer.

\subsection{Community and ecosystem}

As of early 2021, it offers a number of managed services offered include: Hedera Consensus, File, Smart Contracts, Crypotcurrency and Tokenization Service. It has a small but growing community with frequent communication updates.

% Code
The code will only be available as open review for release 1.0. There are no plans to open-source it as part of its governance plan but anyone can raise a proposal to improve the network through the Hedera Improvement Proposal (HIP) process. This will prevent forks from happening as Hedera is the only authorised user of the proprietary technology. 

\subsection{Use cases and council members}

Blockchain 3.0 <CITE> discusses usecases for a third generation DLT such as election, micro-payments, supply chain management.

In the UK, Everyware is currently using Hashgraph to track the supply chain of the COVID-19 vaccine \footnote{https://www.cnbc.com/2021/01/19/uk-hospitals-use-blockchain-to-track-coronavirus-vaccine-temperature.html}. Temperature sensitive vaccines need a tamper-proof system to ensure it's proper delivery.

Eftpos Australia is currently developing the next-generation micropayments technology which may potentially open new ways for Australian businesses and consumers to interact. Use cases are are currently being developed as proof-of-concept which includes sub-cent payments to unblock online paywalls\footnote{https://www.eftposaustralia.com.au/news/eftpos-joins-Hedera-Governing-Council-and-will-run-Aussie-Hedera-network-node}.

Central bank digital currencies\footnote{https://hedera.com/learning/what-is-a-central-bank-digital-currency-cbdc} (CBDC) is another potential use-case. Only a few countries have rolled out their own CBDCs. Although it is speculative which technologies central banks are evaluating, more central banks may follow given the interest and investment in the area. 

Similar to the early days of the internet, the technical innovation of this project may enable more use-cases in the future. Hashgraph's technical capabilities and governance model is an enabler which are not to be overlooked.

% https://www.eftposaustralia.com.au/news/eftpos-joins-Hedera-Governing-Council-and-will-run-Aussie-Hedera-network-node
% cite

% https://www.prnewswire.com/news-releases/everyware-and-hedera-hashgraph-enabling-cold-chain-monitoring-of-covid-19-vaccine-for-nhs-facilities-301209642.html


% https://www.hedera.com/users/adsdax
% https://www.hedera.com/users/coupon-bureau/
