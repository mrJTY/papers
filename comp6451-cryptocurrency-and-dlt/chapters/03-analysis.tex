\section{Current state and analysis}

\subsection{Governance and path to decentralisation}
% Governance
Originally under Swirlds, Hashgraph's ownership was transferred to the Hedera Governing Council. It is a decentralised council of term-limited multinational companies \cite{baird2018hedera}. 

It's decentralised governance is based on Visa's model\footnote{https://hedera.com/council} which ensures that council does what is best for the network. No single company has complete control and any malicious acts will certainly damage a member's reputation. Well-known companies are in the council which includes Google, Boeing, LG, and Eftpos Australia. 

A criticism of Hashgraph is that the network must agree on N, the total number of participants \cite{kauflin2018}. As of early 2021, it is currently on a permissioned system which makes N known. Citics also argue because of it's permissioned nature, it is not true to the roots of a fully decentralised network.

% It is still on a path towards decentralisation while ensuring security along the way.
% https://hedera.com/learning/what-is-hedera-hashgraph

\subsection{Open review, not open source}
The source code will only be available as open review, not open source. However, anyone can raise a proposal to improve the network by raising a Hedera Improvement Proposal (HIP)\footnote{https://github.com/hashgraph/hedera-improvement-proposal}. This governance model will prevent forks similar to what happened to Bitcoin Cash and Ethereum Classic. Hedera is the only authorised body that can use the proprietary Hashgraph technology.

This model can be controversial in the cryptocurrency community which has factions that favour a fully decentralised model. However, one has to look into where mining power has been concentrated in a proof-of-work system, and coin concentration in a proof-of-stake system to fully assess a network's true decentralisation of power.

\subsection{Security, staking and tokenomics}
% Analysis on security
Currently on a path towards decentralisation, Hashgraph's proof-of-stake system will rely on 1/3 or more of the coins are not owned by malicious actors. Hedera has outlined a coin release schedule without compromising the network's security. All coins have already been minted and saved in a treasury which will be released over 15 years\cite{economics2020hedera}. Once the coins are released, it will be extremely difficult for a single actor to accumulate enough coins to compromise the network, as surges in demand will increase the price of a limited supply coin.

Compared with the proof-of-work system used by Bitcoin, proof-of-stake is more environmentally friendly as no mining is involved. It can also be argued to be fairer because it does not favour entities with economies of scale or access to cheap electricity. 

\subsection{Correctness}
Gossip relies on assumptions to be fully robust \cite{alvisi}. Gossip is a well known protocol, but combining it with the concept of virtual voting is novel. As it is a new concept, the algorithm was checked by a Coq system (a formal verification system) which proves that it is aBFT - the highest security possible in a distributed system \cite{coq2018}.

\subsection{Performance analysis}
Depending on the region setup, real world experiments using AWS instances have shown throughputs of 50-000-500,000+ transactions-per-second\cite{baird2018hedera}. It still remains to be battle tested in a permissionless setup, but its current transaction volumes\footnote{https://hedera.com/dashboard} already surpass what Bitcoin and Ethereum can do.

% \subsection{Community and ecosystem}


% As of early 2021, it offers a number of managed services offered include: Hedera Consensus, File, Smart Contracts, Crypotcurrency and Tokenization Service. It has a small but growing community with frequent communication updates.

% Code

% Additionally, Blockhain-based systems still benefit from incumbency and familiarity \cite{khariff2018}.

\subsection{Use cases}

Blockchain 3.0\cite{maesa2020blockchain} discussed use cases for a third generation DLT such as elections, micro-payments, supply chain management.

In the UK, Everyware is currently using Hashgraph to track the supply chain of the COVID-19 vaccine\cite{ryan2021}. Temperature sensitive vaccines need a tamper-proof system to ensure it's proper delivery.

Eftpos Australia is currently developing the next-generation micropayments technology which may potentially open new ways for Australian businesses and consumers to interact. Use cases are are currently being developed as proof-of-concepts which includes sub-cent payments to unblock online paywalls\cite{eftpos2021}.

Central bank digital currencies\footnote{https://hedera.com/learning/what-is-a-central-bank-digital-currency-cbdc} (CBDC) is another potential use-case. Only a few countries have rolled out their own CBDCs. Although it is speculative which technologies central banks are evaluating, more central banks may follow given the interest and investment in the area. 

Similar to the early days of the internet, the technical innovation of this project may enable more use-cases in the future. Hashgraph's technical capabilities and governance model is an enabler which are not to be overlooked.

% https://www.eftposaustralia.com.au/news/eftpos-joins-Hedera-Governing-Council-and-will-run-Aussie-Hedera-network-node
% cite

% https://www.prnewswire.com/news-releases/everyware-and-hedera-hashgraph-enabling-cold-chain-monitoring-of-covid-19-vaccine-for-nhs-facilities-301209642.html


% https://www.hedera.com/users/adsdax
% https://www.hedera.com/users/coupon-bureau/
