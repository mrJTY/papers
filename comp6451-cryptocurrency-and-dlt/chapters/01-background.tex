\section{Background}

To understand Hashgraph, one has to understand an overview of current distributed ledger technologies (DLT). Bitcoin\cite{nakamoto2008peer} is a decentralised digital cryptocurrency that enabled transactions without a central intermediary. It uses a technology called the blockchain for participants to agree on their balances on the ledger. As a proof-of-work, blocks are added to the blockchain by the process of mining. Participants try to come to an agreement on a longer chain with the reasoning that more "work" has been put on it. Ethereum\cite{wood2014ethereum}, dubbed as a second generation DLT, further improved the concept by enabling smart contracts to operate on the blockchain.

While these early generation DLTs rose in popularity, wide-spread mass adoption has been limited due to blockchain's ability to scale. At the time of writing, Bitcoin and Ethereum can only currently do around 3+ transactions-per-second (TPS)\cite{bitcointps} and 12+ TPS respectively\cite{ethereumtps}. This is significantly below what incumbent systems such as Visa can provide\cite{visafactsheet} that powers transactions in a modern economy.

Additionally, most blockchains using proof-of-work can only offer probablistic eventual finality\cite{anceaume2020finality}. Finality is a guarantee that past transactions cannot be changed. With Bitcoin, confirmation generally takes about 10 mins. For larger amounts of money, it is typically recommended to wait for up to 6 blocks or about 60 mins to minimise risk of double spending or transaction reversal\cite{coinmarketconfirm}. This type of finality is only probablistic where a shorter chain may catch up with a longer chain given enough hashing power.
% When a proof-of-work is used to achieve consensus, the rate of which blocks can be added were intentionally made slow so that a leader can decide consensus.

Gossip protocols are known to efficiently broadcast information with high reliability and throughput\cite{birman}. Achieving consensus was historically done by sending votes across the network\cite{berman1989towards}. Sending votes across is an expensive operation that has not been reliably implemented in real-world conditions.

Hedera Hashgraph\cite{baird2016} combined a gossip protocol with the concept of virtual voting to achieve consensus in a radically different way than a blockchain. The network can reach consensus in a more efficient manner than the current proof-of-work process of mining. As a result, it provides a fast, high-throughput, fair-ordering, finality, and asynchronous Byzantine Fault Tolerant (aBFT) alternative to blockchain solutions.
