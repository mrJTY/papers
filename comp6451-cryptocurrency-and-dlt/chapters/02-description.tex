\section{Technical innovation of the protocol}

% insert figure

Hashgraph uses cryptographic hashes and digital signatures to securely spread events across the network. Each member will eventually have a consistent hashgraph which they can efficienlty query to do virtual voting without sending votes across the network.

To achieve consensus, consider first the the Byzantine Generals Theorem \cite{shostak1982byzantine}. It states that we cannot reliably reach consensus if more than 1/3 of participants are malicious. ABFT is considered to be the gold standard in security as it can be resistant to firewalls, DDoS attacks, and messages get lost or altered.

Hashgraph was proven to be aBFT with its key pillar of the Strongly Seeing Lemma \cite{baird2016}. Suppose Mallory can cheat by forking an event on her own graph. She could double spend her coins by gossiping two different events (x \& y) to different members of the network. The Strongly Seeing Lemma states that the forked event will not be strongly seen by other members. A proof by contradiction shows that if 2/3 strongly sees x and 2/3 also strongly sees y, then this is impossible if less than 1/3 are malicious.

% To prove how it works, let us define what strongly seeing an event means.

% The Strongly Seeing Theorem <CITE> is the key pillar of how hashgraph works. The proof is by contradiction. Suppose Mallory wants to cheat the system by forking an event. (This is a condensed summary of the proof, see the whitepaper for the full proof) Alice's local hashgraph reads 2n/3 x strongly sees w and another 2n/3 strongly sees y, then the intersection must be >1/3. The assumption is no more than 1/3 is dishonest so there must be an honest member in between. However, that honest member cannot strongly see both events because x and y are fork and breaks the definition of strongly seeing. This is a contradiction therefore the assumption that x strongly sees w is false. Mallory will not be able to cheat if no more than 1/3 of the nodes are malicious.
