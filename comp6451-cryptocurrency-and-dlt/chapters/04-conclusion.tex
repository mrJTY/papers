\section{Conclusion and final remarks}

In a saturated market of cryptocurrencies with calls to "hodl to the moon", there is a lot of noise and hype in this rapidly changing environment. Hashgraph stands out in a number of ways. Its technical innovation offers efficiency that can enable DLTs to scale. Its governance structure can enable mainstream enterprise adoption with a path towards decentralisation. It is rare to find a cryptocurrency with these features that breaks away from the blockchain, let alone a formal verification by a Coq system that proves it is aBFT - the strongest level of security in a distributed system.

It's not so often that a technology can be disruptive and make a generational leap. Bitcoin and Ethereum may have paved the way for first and second generation DLTs, Hedera Hashgraph has the potential to take the lead of the third generation DLTs. It has a lot of potential on paper and on real world use cases. While there are tangible use cases happening today, there are still more items to be developed in the roadmap and more council members to be added. Similar to the early days of the internet, Hashgraph's technical capabilities and governance model may become a potential enabler for more use cases in the future.
