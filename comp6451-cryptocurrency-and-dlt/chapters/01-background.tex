\section{Background}

Bitcoin\cite{nakamoto2008peer} is a decentralised digital cryptocurrency that enabled transactions to be done without a central intermediary. It uses a distributed ledger technology (DLT) called the blockchain for participants to agree on their balances on the ledger. Ethereum\cite{wood2014ethereum}, a second generation DLT, further improved the concept by enabling smart contracts to operate on the blockchain. While these early generation DLTs rose in popularity, wide-spread mass adoption has been limited due to blockchain's ability to scale. 

% When a proof-of-work is used to achieve consensus, the rate of which blocks can be added were intentionally made slow so that a leader can decide consensus.

Gossip protocols are known to efficiently broadcast information with high reliability and throughput\cite{birman}. Achieving consensus was historically done by sending votes across the network\cite{berman1989towards}. Sending votes across is an expensive operation that has not been reliably implemented in real-world conditions.

Hedera Hashgraph\cite{baird2016} combined a gossip protocol with the concept of virtual voting. This enabled the network to reach consensus in a more efficient manner than the proof-of-work system used in current blockchain implementations. As a result, it provides a fast, high-throughput, fair-ordering, and asynchronous Byzantine Fault Tolerant (aBFT) alternative to blockchain solutions.
